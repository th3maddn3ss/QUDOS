
% QUDOS Whitepaper
% This LaTeX document provides an overview of the QUDOS project and serves as the
% primary white‑paper for dissemination.  It is intended for publication on
% Zenodo under the authorship of Devin Lavrisha with ChatGPT listed as a
% research assistant.

\documentclass[11pt]{article}
\usepackage[a4paper,margin=1in]{geometry}
\usepackage{amsmath}
\usepackage{hyperref}
\usepackage{graphicx}
\usepackage{float}
\usepackage{booktabs}
\title{QUDOS: A Reduced Instruction Set Approach to Quantum Qudit Control}
\author{Devin\\,Lavrisha \\\\ ChatGPT (research assistant)}
\date{\today}

\begin{document}
\maketitle

\begin{abstract}
QUDOS (Quantum Universal Digital Operator Scheduler) is a proof‑of‑concept
project implementing a reduced instruction set computer (RISC) style
instruction set for controlling qudit hardware.  This document describes
the motivation for using multi‑level qudits over conventional two‑level
qubits, outlines the design of a compact instruction format and its
translation into physical waveform schedules, and provides an audit of the
repository accompanying this white‑paper.  The included example program
demonstrates the assembly of four high‑level instructions (controlled
phase, controlled rotation, phase update and rotation) into 32‑bit words
and the synthesis of these words into a timed schedule of Gaussian
pulses.  Citations are provided for statements of fact and background
concepts.
\end{abstract}

\section{Introduction}
Quantum information is usually encoded in two–level systems known as
\emph{qubits}, but physical hardware often possesses more than two
energy levels.  The ability to exploit higher‑dimensional Hilbert
spaces leads to \emph{qudits}, multi‑level generalizations of qubits.
The Entropy Quantum Computing (EQC) paradigm notes that qudits are
``units of quantum information taking more than 2 possible states,'' and
that combining superposition with the ability to represent more than two
levels makes computing with qudits powerful【303900817418565†L31-L35】.
Another source observes that a qudit can exist in $d$ states while a
qubit is limited to two; common examples include the qubit ($d=2$),
qutrit ($d=3$) and ququart ($d=4$)【125270124644100†L29-L38】.  Mathematically
these states live in a $d$‑dimensional Hilbert space and are described
by vectors of complex amplitudes that satisfy a normalization
constraint【125270124644100†L39-L50】.  Because qudits have more degrees of
freedom, the same problem can often be encoded with fewer physical
particles; they can also natively express integer and continuous
variables, allowing problems beyond binary optimisation to be addressed
【303900817418565†L55-L69】.

Control of a quantum processor relies on accurately sequencing and
shaping analog waveforms.  In many physical platforms the envelope of
each control pulse is well approximated by a Gaussian function, i.e.,
a pulse whose intensity profile has a Gaussian shape【966124329252690†L29-L31】.
This white‑paper presents a minimal instruction set for qudit
operations and an associated compiler and scheduler that translate
high‑level instructions into fixed‑width words and ultimately into a
timed list of Gaussian pulses and frame updates.

\section{Background}

\subsection{Qudits and Superposition}
In quantum computing the fundamental resource is superposition: a
quantum state can exist in a linear combination of basis states at
once.  For qubits superposition produces a two‑term sum, whereas a
qudit with dimension $d$ can occupy any superposition of $d$ basis
states【303900817418565†L75-L79】.  The EQC literature emphasises that
qudits inherit all capabilities of qubits and add new ones; any qubit
computation can be expressed in terms of qudits, but qudits can also
represent integer variables and approximate continuous quantities
natively【303900817418565†L45-L69】.  Because of these properties, qudits
are being explored in platforms ranging from superconducting circuits to
optical multi‑mode encodings【125270124644100†L75-L118】.

\subsection{Reduced Instruction Set Computers}
The classical RISC philosophy inspired the design of QUDOS.  A reduced
instruction set computer simplifies the individual instructions given
to the processor; compared to a complex instruction set computer
(CISC), more instructions may be required to accomplish the same task,
but each instruction is simpler and can execute more quickly
【40348738344627†L188-L205】.  RISC designs typically provide many high‑speed
registers and separate memory load/store instructions, leading to
predictable instruction timings and efficient pipelining【40348738344627†L199-L206】.
These ideas are adapted here to the domain of quantum control: a small
set of instructions carry the essential information needed to generate
waveforms while leaving higher‑level sequencing to the compiler.

\section{Instruction Format and Assembly}
\subsection{Instruction Set}
The QUDOS instruction set defines four operations:
\begin{itemize}
  \item \textbf{\texttt{QCPZ}} – Controlled Phase on target register
    $r_{\mathrm{t}}$ controlled by register $r_{\mathrm{c}}$ at level $k$ with
    phase angle $\theta$.
  \item \textbf{\texttt{QCRT}} – Controlled Rotation on target register
    $r_{\mathrm{t}}$ controlled by register $r_{\mathrm{c}}$ with indices $i,j$ and
    level $k$ and rotation angle $\theta$.
  \item \textbf{\texttt{QPHZ}} – Phase update on register
    $r_{\mathrm{t}}$ at level $k$ by angle $\theta$.
  \item \textbf{\texttt{QROT}} – Rotation of register $r_{\mathrm{t}}$ around
    axes $i,j$ by angle $\theta$.
\end{itemize}

Each instruction is encoded into a 32‑bit word.  Let $\mathrm{op}$ be
the 4‑bit opcode, $r_{\mathrm{t}}$ the 4‑bit target register, $r_{\mathrm{c}}$
the 4‑bit control register, $i,j,k$ four‑bit indices, and let
$\theta$ be the rotation or phase angle in radians.  The lower eight
bits store a fixed‑point representation of $\theta$ obtained via
\begin{equation}
  t_{\mathrm{f}} = \bigl\lfloor \theta\,S \rceil \bigr\_8,
\end{equation}
where $S$ is the \texttt{theta\_fixed\_scale} parameter provided in the
placement file and $\lfloor \cdot \rceil\_8$ denotes rounding to the
nearest integer and truncation to eight bits.  The complete word is
assembled by bit–shifting the fields into place:
\begin{equation}
  w_{32} = (\mathrm{op} \ll 28)\;|\;(r_{\mathrm{t}} \ll 24)\;|\;
  (r_{\mathrm{c}} \ll 20)\;|\;(i \ll 16)\;|\;(j \ll 12)\;|
  (k \ll 8)\;|\;t_{\mathrm{f}}.
  \label{eq:pack}
\end{equation}
Algorithm~\ref{alg:assemble} summarizes the assembly process.

\begin{figure}[H]
\centering
\begin{minipage}{0.95\linewidth}
\begin{footnotesize}
\begin{tabbing}
\textbf{Algorithm}\quad\=\textsc{AssembleLines}(\textit{lines},\textit{placement})\\
\textbf{Input:}\>List of instruction strings; placement dictionary\\
\textbf{Output:}\>List of 32‑bit words and parsed operations\\
\textbf{1.}\>Load encoding table and scale $S$ from placement file.\\
\textbf{2.}\>Initialize empty lists for words and operation records.\\
\textbf{3.}\>For each line $\ell$ in \textit{lines}:\\
\>\textbf{a.}\>Skip blank lines or comments.\\
\>\textbf{b.}\>Tokenize $\ell$; determine mnemonic and parse
parameters.\\
\>\textbf{c.}\>Compute $t_{\mathrm{f}}=\lfloor \theta S \rceil\_8$.\\
\>\textbf{d.}\>Compute $w_{32}$ using Eq.~(\ref{eq:pack}).\\
\>\textbf{e.}\>Append $w_{32}$ to words and append the parsed
operation to records.\\
\textbf{4.}\>Return lists of words and operations.
\end{tabbing}
\end{footnotesize}
\end{minipage}
\end{figure}

\subsection{Example Assembly}
Consider the program
\begin{verbatim}
QCPZ 0, 1, 1, 2, 0.7rad
QCRT 0, 1, 1, 2, 3, 0.4rad
QPHZ 1, 2, 15deg
QROT 1, 1, 2, 10deg
\end{verbatim}
running on the placement described in Section~\ref{sec:placement}.  The
assembler converts each angle specified in degrees or radians into
radians, multiplies by the scale $S\approx81.4873$ from the YAML
placement, rounds to the nearest integer and keeps the lower eight
bits.  Table~\ref{tab:listing} lists the resulting words and parsed
parameters.  The words expressed in hexadecimal match those produced by
the provided implementation.

\begin{table}[H]
\centering
\caption{Example program listing.\label{tab:listing}}
\begin{tabular}{@{}llllllll@{}}
\toprule
Index & Word$_{16}$ & Kind & $r_t$ & $r_c$ & $i$ & $j$ & $k$ & $\theta$ (rad)\\
\midrule
0 & 0x31000139 & QCPZ & 1 & 0 & – & – & 1 & 0.7\\
1 & 0x41023121 & QCRT & 1 & 0 & 2 & 3 & 1 & 0.4\\
2 & 0x11000215 & QPHZ & 1 & – & – & – & 2 & 0.2618\\
3 & 0x2101200e & QROT & 1 & – & 1 & 2 & – & 0.1745\\
\bottomrule
\end{tabular}
\end{table}

\section{Waveform Synthesis}
After assembly, the sequence of parsed operations is converted into a
time–ordered list of control pulses.  Each pulse has start time
$t_{0}$, end time $t_{1}$, channel identifier, shape and additional
parameters.  The scheduler uses three pieces of hardware metadata: the
time step $\mathrm{dt\_ns}$ in nanoseconds; the Gaussian pulse
envelope parameters (standard deviation $\sigma$ and total length
$L$); and a calibration constant $R$ describing the scaling between
rotation angles and pulse amplitudes.  Algorithm~\ref{alg:sched}
performs the synthesis.

\begin{figure}[H]
\centering
\begin{minipage}{0.95\linewidth}
\begin{footnotesize}
\begin{tabbing}
\textbf{Algorithm}\quad\=\textsc{SynthQROTZSchedule}(\textit{ops},\textit{placement})\\
\textbf{Input:}\>List of parsed operations and placement dictionary\\
\textbf{Output:}\>Time‑ordered list of pulse descriptors\\
\textbf{1.}\>Initialize time $t=0$~ns and empty schedule.\\
\textbf{2.}\>For each operation $o$ in \textit{ops}:\\
\>\textbf{a.}\>Let $r_{\mathrm{t}}$ be the target register.  Map it to its
 drive channel $c$ via the register mapping.\\
\>\textbf{b.}\>If $o$ is a rotation (\texttt{QROT} or \texttt{QCRT}):\\
\>\quad i.\>Compute pulse duration $d=L$ (Gaussian length).\\
\>\quad ii.\>Compute amplitude $A = R\,|\theta|$.\\
\>\quad iii.\>Look up frequency $f$ from the appropriate level index.\\
\>\quad iv.\>Append $\{t, t+d, c, \text{gaussian}, \sigma, A, f\}$ to schedule.\\
\>\quad v.\>Update $t \leftarrow t + d$.\\
\>\textbf{c.}\>If $o$ is a phase update (\texttt{QPHZ} or \texttt{QCPZ}):\\
\>\quad i.\>Append $\{t, t, c, \text{frame\_update}, \phi\}$ to schedule.\\
\>\textbf{d.}\>Increment $t$ by $\mathrm{dt\_ns}$.\\
\textbf{3.}\>Return schedule.\\
\end{tabbing}
\end{footnotesize}
\end{minipage}
\end{figure}

Applying the scheduler to the example program using $\mathrm{dt\_ns}=0.25$~ns,
Gaussian envelope length $L=6$~ns, $\sigma=1.5$~ns and rotation scale
$R=1$ produces the schedule listed in Table~\ref{tab:schedule}.  The
second and fourth rows describe Gaussian pulses applied to drive
channel $\texttt{d1}$ with amplitudes proportional to the rotation
angle; the other rows describe instantaneous frame updates.

\begin{table}[H]
\centering
\caption{Synthesised waveform schedule for the example program.
\label{tab:schedule}}
\begin{tabular}{@{}lllllll@{}}
\toprule
$t_0$ (ns) & $t_1$ (ns) & Channel & Shape & $\sigma$ (ns) & Amp & Frequency (Hz)\\
\midrule
0.0 & 0.0 & d1 & frame\_update & – & – & –\\
0.25 & 6.25 & d1 & gaussian & 1.5 & 0.4 & $4.94\times10^9$\\
6.5 & 6.5 & d1 & frame\_update & – & – & –\\
6.75 & 12.75 & d1 & gaussian & 1.5 & 0.1745 & $4.92\times10^9$\\
\bottomrule
\end{tabular}
\end{table}

\section{Repository Audit and Improvements}
The accompanying repository contains three primary modules:
\begin{itemize}
  \item \texttt{quit\_risc\_asm\_yaml.py}: Implements the assembly
    algorithm.  It reads a YAML or JSON placement file, parses
    high‑level instructions, converts angles to radians, computes the
    fixed‑point field using the provided scale and packs the fields
    according to Eq.~(\ref{eq:pack}).  The implementation supports
    specifying angles in degrees (suffix \texttt{deg}) or radians
    (suffix \texttt{rad}) and uses a fallback JSON placement if YAML
    parsing is unavailable.  The bit masks and shifts are clearly
    implemented and match the description above.
  \item \texttt{waveform\_synth.py}: Provides the synthesis algorithm
    summarised in Algorithm~\ref{alg:sched}.  For rotation operations it
    computes the pulse amplitude $A=R\,|\theta|$ and reads the
    appropriate frequency from the register mapping.  For phase
    operations it issues a zero‑duration frame update.  The code
    increments the global time by $\mathrm{dt\_ns}$ after every
    instruction, ensuring correct spacing between pulses.
  \item \texttt{assemble\_and\_synthesize.py}: A reproducible experiment
    that loads a placement, assembles a small demonstration program
    and writes four reports: a JSON file listing the decimal and
    hexadecimal words, a CSV program listing, and both CSV and JSON
    versions of the waveform schedule.  Running this script within the
    repository reproduces the tables shown above and thereby verifies
    the correctness of the implementation.
\end{itemize}

\paragraph{Placement files.}  The repository distributes both YAML
and JSON versions of the placement description.  Minor differences
exist between them—for example the JSON lists a slightly different
\texttt{theta\_fixed\_scale} value (81.1690 vs.~81.4873).  For
consistency we recommend using the YAML version as the canonical
placement, or documenting the difference and its implications.  Both
files specify the hardware clock frequency, available arbitrary
waveform generators (AWGs), register mappings with level frequencies,
timing parameters, waveform envelope parameters and calibration
constants.

\paragraph{Recommendations for Zenodo.}  To prepare this project for a
Zenodo deposit we added a \texttt{README.md} explaining the structure
and purpose of each file, created this white‑paper (in \LaTeX) and a
separate PDF documenting the mathematical derivations.  Additional
improvements could include:
\begin{itemize}
  \item Providing a license file (e.g., MIT or Apache‑2.0) to clarify
    reuse of the code.
  \item Adding unit tests to verify correct assembly and scheduling for
    a variety of instructions and placement configurations.
  \item Including a \texttt{requirements.txt} file listing Python
    dependencies and supported versions.
  \item Publishing a DOI by depositing the release on Zenodo with
    metadata listing the author (Devin Lavrisha) and acknowledging
    ChatGPT as a research assistant.
\end{itemize}

\section{Conclusion}
This white‑paper introduced QUDOS, a minimalist RISC‑inspired
instruction set for controlling qudit hardware.  By encoding
instructions into 32‑bit words and translating them into timed lists
of Gaussian pulses and frame updates, QUDOS provides a bridge between
high‑level algorithmic descriptions and the low‑level analogue control
required by quantum hardware.  The example program assembled and
synthesised here demonstrates the end‑to‑end pipeline: a sequence of
operations is encoded into machine words and then scheduled into
control pulses with amplitudes, frequencies and timing derived from
hardware parameters.  The repository audit highlighted the
completeness of the current implementation and suggested practical
steps for enhancing reproducibility and preparing the project for a
Zenodo release.

\end{document}
